\section{Quick start}

We illustrate Pylightnix principles by showing how to use it as a Nix-like build
system. We will define two \textbf{stages} required to build
\href{https://www.gnu.org/software/hello/}{GNU Hello} application.

Note that despite using similar approaches, Pylightnix doesn't aim to be a
direct Nix replacement. It lacks many features you would probably expect of a
OS-level package management tool. It has no multiprocessing support neither it
has a build isolation. In contrast to Nix, Pylightnix is a bare-bone library you
could easily integrate into other packages to handle project-level data
management tasks.

\subsection{Defining a Stage}

The following Python function defines a Pylightnix \textbf{Stage} entity named
\texttt{stage\_fetch}. In its body we use a pre-defined \texttt{fetchurl}
finction which binds some name, URL, and the SHA256 hash string with a built-in
algorithm calling \texttt{Wget} and \texttt{Aunpack} tools to download GNU Hello
sources from the Internet.

\begin{pythontexcode}
from pylightnix import (Manager, DRef, RRef, fetchurl)

hello_version = '2.10'

def stage_fetch(m:Manager)->DRef: # DRef \label{DREF}
  return fetchurl(m,
    name='hello-src',
    url=f'http://ftp.gnu.org/gnu/hello/hello-{hello_version}.tar.gz',
    sha256='31e066137a962676e89f69d1b65382de95a7ef7d914b8cb956f41ea72e0f516b')
\end{pythontexcode}

\begin{itemize}

  \item Indeed, Pylightnix stages are Python functions. By calling a stage
    function we register or \textbf{instantiate} this stage, that is, we bind
    some realization algorithm with its input arguments here called
    prerequisites.

  \item Stage function takes a dependency-resolution \textbf{Manager} context as
    its first arguments. The Manager holds the information about instantiated
    stages. Users normally shouldn't operate on Managers or call stage functions
    directly. Instead, they are to pass the top-level stage to one of Pylightnix
    top-level \texttt{instantiate-} functions.

  \item The return value of a stage function is a \textbf{derivation reference}.
    The primary role of derivation references is to introduce new dependencies
    between stages. By including a derivation reference of one stage into
    prerequisites of another stage we say that the second stage depends on the
    first stage.


\end{itemize}

\pagebreak
\subsection{Accessing the Data}

The primary method of accessing stage data is to call \texttt{realize1} function,
passing the \texttt{instantiated} stage as its argument. Pylightnix either runs
the algorithm associated with a stage or re-uses existing realization. In both
cases, a \textbf{realization reference} handle is returned. We then use a
\texttt{Lens} helper which is a 'swiss army knife' for navigating through the
Pylihgtnix storage.

\begin{pythontexcode}
from pylightnix import instantiate, realize1, mklens
from os.path import join, isdir

fetch_rref:RRef = realize1(instantiate(stage_fetch))  # RRef \label{RREF}
print(fetch_rref)

print(mklens(fetch_rref).val)      # Interpret as Python value
print(mklens(fetch_rref).rref)     # Interpret as RRef (same result)
print(mklens(fetch_rref).dref)     # Interpret as DRef
print(mklens(fetch_rref).syspath)  # Interpret as a filesystem path
assert isdir(join(mklens(fetch_rref).syspath,
             f"hello-{hello_version}"))  # SRC \label{SRC}
print(mklens(fetch_rref).name.val) # Access prerequisite 'name' variable
                                   # and interpret it as a Python value
print(mklens(fetch_rref).url.val)  # Same for 'url' variable
print(mklens(fetch_rref).sha256.val)  # Same for 'sha256' variable
\end{pythontexcode}

Output:

\mysmallstdout

\begin{itemize}

  \item At line \ref{RREF} we \texttt{instantiate} and \texttt{realize1} our
    custom stage to get its realization reference.

  \item Realization reference is a string identifier which could be converted to
    a local filesystem path within the storage. This path contains the sources
    of GNU Hello application. At line \ref{SRC} we check that this path is
    valid.

\end{itemize}

\pagebreak
\subsection{Defining a Custom Stage}

In this section we define a custom Pylightnix stage named \texttt{stage\_build}.
In its algoritm we provide instructions how to build the GNU Hello application
out of its sources.

\begin{pythontexcode}
from tempfile import TemporaryDirectory
from shutil import copytree
from os import getcwd, chdir, system
from pylightnix import (Config, Path, Build, mkconfig, mkdrv,
  build_wrapper, match_only, dirrw, selfref,
  __version__ as pylightnix_version)

def stage_build(m:Manager)->DRef:                           # ST \label{ST}
  def _config()->Config:                                    # CFG \label{CFG}
    name:str = 'hello-bin'
    fetchref:DRef = stage_fetch(m)                          # DEP \label{DEP}
    src:RefPath = [fetchref, f'hello-{hello_version}']      # RP1 \label{RP1}
    bin:RefPath = [selfref, 'usr', 'bin', 'hello']          # RP2 \label{RP2}
    built_with = pylightnix_version                         # VER \label{VER}
    return mkconfig(locals())                               # LOC \label{LOC}
  def _realize(b:Build)->None:                              # RE \label{RE}
    with TemporaryDirectory() as tmp:
      copytree(mklens(b).src.syspath, join(tmp,'src'))      # LE \label{LE}
      dirrw(Path(join(tmp,'src')))
      cwd = getcwd()
      try:
        print(f"Building {mklens(b).name.val}")
        chdir(join(tmp,'src'))
        system(f'./configure --prefix=/usr')
        system(f'make')
        system(f'make install DESTDIR={mklens(b).syspath}')
      finally:
        chdir(cwd)
  return mkdrv(m, config=_config(),
                  matcher=match_only(),
                  realizer=build_wrapper(_realize))          # DRV \label{DRV}

build_rref:RRef = realize1(instantiate(stage_build))
print(build_rref)
\end{pythontexcode}

Output:

\mysmallstdout

\begin{itemize}
  \item At line \ref{ST} we define a stage function.

  \item It is more convenient to start reading the stage code from the end.  At
    line \ref{DRV} we call the \textbf{mkdrv} stage constructor. The function
    has the following arguments:

    \begin{itemize}

      \item \textbf{Manager} utility object. Managers describe dependency
        resolution contexts and have no special meaning for end-users. We just
        pass the object we obtain as a stage function argument down to the stage
        constructor.

      \item \textbf{Config} dictionary-like object. Configs represent stage's
        prerequisites. Configs may contain any JSON-serializable values of
        standard Python: strings, numbers, booleans, lists or other dictionaries
        (but no tuples).

      \item \textbf{Matcher} function. Matchers let Pylightnix decide whether to
        run realizer or to re-use an existing object or objects.

      \item \textbf{Realizer} function. Realizers specify how to create a new
        stage object out of its prerequisites.

    \end{itemize}

    In this example we expect to get exactly one outcome. We use a pre-defined
    \textbf{match\_only} matcher which ensures that our stage has only one
    realization.

  \item At line \ref{CFG} we define our stage's build prerequisites. For
    convenience we user a Python lifehack - call \texttt{local()} to convert
    local variables of \texttt{\_config()} into a dictionary.

  \item What to put into prerequisites? Basically, anything that we think may
    affect the results of stage's realization. For this example we choose the
    following fields:

    \begin{itemize}

      \item \textbf{Name}. In Pylightnix, names are optinoal fields with a special
        meaning. If specified, Pylightnix appends it to derivation and
        realization references as a visual hint.

      \item \textbf{Dependency derivation references}. In this example, we let
        Pylightnix know that \texttt{stage\_build} depends on
        \texttt{stage\_fetch} by mentioning \texttt{fetch\_ref} at line
        \ref{DEP}. Later we use a lens at line \ref{LE} to access the
        \texttt{stage\_fetch}'s output data.

      \item \textbf{RefPaths}. At line \ref{RP1} we use Python list to define a
        relative path, pointing to a folder inside the dependency's realization.
        Realization objects may not yet exist at the configuration pass, but
        they become accessible during the realization.

      \item At line \ref{RP2} we also define a \textbf{Promise Path}. Pylightnix
        will replace promise markers with a derivation references of the stage
        being built. After the completion of realization algorithm, it will also
        check that said objects have appeared in the filesystem.

      \item We may add anything we want Pylightnix to track for us. At line
        \ref{VER} we ask it to remember its own version. Thanks to this
        assignment, Pylightnix re-runs the realization algorithm on every API
        change and by this helps the author to keep this manual up-to-date.
        Similarly, we could add a Git hash of an important source, sha256
        checksum of a machine learning dataset or other task-specific
        information.

    \end{itemize}

  \item With \texttt{mklens} we could explore dependencies in-depth. The
    definition at line \ref{DEP} allows us to read the parent's prerequisites as
    follows:

    \begin{pythontexcode}
    print(mklens(build_rref).fetchref.rref)
    print(mklens(build_rref).fetchref.url.val)
    \end{pythontexcode}

    Output:

    \mysmallstdout

\end{itemize}

\pagebreak
\subsection{Execute}

Finally, we locate the GNU Hello binary and run it.

\begin{pythontexcode}
from subprocess import run, PIPE
print(run([mklens(build_rref).bin.syspath], stdout=PIPE).stdout.decode('utf-8'))
\end{pythontexcode}

Output:

\mystdout


