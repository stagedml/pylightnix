\section{Install}

Pylightnix could be installed either by running Pip package manager as usual or
by bundling the package from source. Since we are in deep betas, you probably
should prefer source-based installation. We will also need
\href{https://curl.se/}{Curl} and \href{https://www.nongnu.org/atool/}{ATool}
system packages later in this manual.

\subsection{Install with Pip}

\begin{shellcode}
$ pip3 install pylightnix
\end{shellcode}

\subsection{Build from source}

\begin{enumerate}
  \item Clone the repo
    \begin{shellcode}
    $ git clone https://github.com/stagedml/pylightnix
    $ cd pylightnix
    \end{shellcode}
  \item Either
    \begin{itemize}
      \item Set the \texttt{PYTHONPATH} environment variable to import
        Pylightnix in-place.
        \begin{shellcode}
        $ export PYTHONPATH="`pwd`/src:$PYTHONPATH"
        \end{shellcode}
      \item Build and install the wheel package.
        \begin{shellcode}
        $ make wheel
        $ sudo -H make install
        \end{shellcode}
      \item See the Development section of the Manual for guidance on operating
        a development environment.
    \end{itemize}
  % \item (Optional) Run the tests.
  % \item (Optional) Make docs
  % \item (Optional) Build the demos
\end{enumerate}

\subsection{Install the recommended system packages}

Pylightnix utilities rely on Curl and ATool system packages. Installing them
is highly advised.

\begin{shellcode}
$ apt-get install -y curl atool      # Use your system's package manager here!
...
$ curl --version | head -n1
curl 7.70.0
$ aunpack --version | head -n1
atool 0.39.0
\end{shellcode}

\subsection{Make sure GCC and GNU Autotools are available}

Some demonstration code from this manual also uses GCC and GNU Autotools. Make
sure they are installed.


