\section{Quick start}

We illustrate Pylightnix principles by showing how to use it for making a mock
data science task. We will follow the
\href{https://docs.scipy.org/doc/scipy/reference/generated/scipy.optimize.dual_annealing.html}{SciPy
annealing example} but imagine that we additionally want to save some intermediate results
and also that there are several people working on this task.

We start by importing the required Python modules.

\begin{pythontexcode}
import numpy as np
import scipy.optimize as o
import matplotlib.pyplot as plt

from pylightnix import *
\end{pythontexcode}

\subsection{The problem}

The problem begins with defining a parametric function \texttt{f} for which we
need to find an approximate minimum by running the simulated annealing
algorithm.

\begin{pythontexcode}
def f(z, *params):
  x, y = z
  a, b, c, d, e, f, g, h, i, j, k, l, scale = params
  return (a * x**2 + b * x * y + c * y**2 + d*x + e*y + f) + \
         (-g*np.exp(-((x-h)**2 + (y-i)**2) / scale)) + \
         (-j*np.exp(-((x-k)**2 + (y-l)**2) / scale))
\end{pythontexcode}

Suppose we are given the parameters for \texttt{f} and our engineering goals
are: (a) to save the parameters (b) to run the annealing simulation and (c) to
make a visual plot of the result.

\subsection{Pylightnix stages}

Pylightnix offers the \pref{Stage}{pylightnix.types.Stage} mechanism to organize
non-deterministic Python functions into a directed acycilc graph, where each
node has the following attributes:

\begin{itemize}
  \item config - a set of user-defined pre-requisite parameters and references
    to parent nodes.
  \item matcher - either built-in or user-defined Python function to filter
    subsets of possible non-deterministic results.
  \item realizizer - a user-defined function producing one or many sets of
    possibly non-deterministic results.
\end{itemize}

\subsubsection{Parameter stage}

The shortest way of defining Stages is to wrap a future-realizer function with
the \pref{autostage}{pylightnix.deco.autostage} decorator.

\begin{pythontexcode}
@autostage(name='params',
           out=[selfref, "params.npy"])
def stage_params(build:Build, name:str, out:str)->DRef:
  np.save(out, (2, 3, 7, 8, 9, 10, 44, -1, 2, 26, 1, -2, 0.5))
\end{pythontexcode}

The decorator accepts an arbitrary number of parameters which together form the
set of pre-requisites for our stage. The decorator passes most of them to the
wrapped function, but it changes some types according to the rules explained
below and in the decorator's reference.

In the above code the \texttt{selfref} list of strings named \texttt{out}
is automatically translated to a system path with the same name. Upon the
realizer's completion, this path should contain a file or a folder. Pylightnix
doesn't check its contents in any way, but not having this file at all is an
error.

The decorator also adds the build context object named
\pref{Build}{pylightnix.types.Build} (which is not used here) and changes the
type of the function result from \texttt{None} to
\pref{DRef}{pylightnix.types.DRef} which stands for "Derivation Reference".
Derivation references may be used to either run corresponding stages or to link
them to other dependant stages.

In this example we are pretending to receive input parameters from a
third-party. We could just as well have downloaded them from the Internet using
a pre-defined stage \pref{fetchurl}{pylightnix.stages.fetchurl2}.

\subsubsection{Annealing stage}

\begin{pythontexcode}

@autostage(name='anneal',
           trace_xs=[selfref, 'tracex.npy'],
           trace_fs=[selfref, 'tracef.npy'],
           out=[selfref, 'result.npy'],
           sourcedeps=[f])
def stage_anneal(build:Build, name,
                 trace_xs:str, trace_fs:str, out:str,
                 ref_params)->DRef:
  params = np.load(ref_params.out)
  xs = []; fs = []
  def _trace(x, f, ctx):
    nonlocal xs, fs
    xs.append(x.tolist())
    fs.append(f)
  res = o.dual_annealing(f, [[-10,10],[-10,10]],
                         x0=[2.,2.], args=params,
                         maxiter=500, callback=_trace)
  np.save(trace_xs, np.array(xs))
  np.save(trace_fs, np.array(fs))
  np.save(out, res['x'])
\end{pythontexcode}

In the above stage we see another special decorator parameter named
\texttt{sourcedeps}. This parameter accepts a list of arbitrary Python objects
whose source we want Pylightnix to include into the set of pre-requisites. Here
we want Pylightnix to re-build the stage every time the \texttt{f}'s sources
change.

Also note \texttt{ref\_params} parameter of the realizer which is not included
into the decorator. This parameter will be provided additionally as a
\texttt{DRef} reference. Yes, this is how we specify the dependencies between
stages. The decorator translates the reference into a Python object which has
all parent pre-requisites visible as attributes. Pylightnix guarantees that all
selfref paths of the dependency (like \texttt{out} here) do point to valid
file-system objects at the time of stage execution.

\subsubsection{Plotting stage}

\begin{pythontexcode}
@autostage(name='plot', out=[selfref, 'plot.png'])
def stage_plot(build:Build, name:str, out:str, ref_anneal:dict)->DRef:
  xs=np.load(ref_anneal.trace_xs)
  fs=np.load(ref_anneal.trace_fs)
  res=np.load(ref_anneal.out)
  plt.figure()
  plt.title(f"Min {fs[-1]}, found at {res}")
  plt.plot(range(len(fs)),fs)
  plt.grid(True)
  plt.savefig(out)
\end{pythontexcode}

\subsubsection{Running the experiment}

To run the experiment we link stages together by calling them under the control
of dependency \pref{Registry}{pylightnix.types.Registry}.  We pass the last
reference to \texttt{instantiate} and \texttt{realize1} functions. The realize
run the realizares in the right order and returns a second kind of reference
called \pref{realization reference}{pylightnix.types.RRef}. This new reference
points to both the original derivation and to a particular set of build results.

\begin{pythontexcode}
with current_registry(Registry()):
  dref_params = stage_params()
  dref_anneal = stage_anneal(ref_params=dref_params)
  dref_plot = stage_plot(ref_anneal=dref_anneal)
  rref = realize1(instantiate(dref_plot))
  print(rref)

  assert isdref(dref_plot)
  assert isrref(rref)
\end{pythontexcode}

\mystdout

\subsection{Nested stages}

In the previous sections we used a global
\pref{Registry}{pylightnix.types.Registry} object to register stages. Another
way of combining stages together is to wrap them with an umbrella stage
describing the whole experiment.

\begin{pythontexcode}
def stage_all(r:Registry)->DRef:
  ds = stage_params(r)
  cl = stage_anneal(r,ref_params=ds)
  vis = stage_plot(r,ref_anneal=cl)
  return vis
\end{pythontexcode}

As you can see, the umbrella stage is a regular Python function which takes a
\texttt{Registry} and returns a \texttt{DRef}.

Now we may pass the top-level stage directly to \texttt{instantiate}
which would call it with a local disposable Registry.

\begin{pythontexcode}
rref_2 = realize1(instantiate(stage_all))
assert rref_2==rref
print(rref_2)
\end{pythontexcode}

\mystdout


\subsection{Collaborative work}

Alice and Bob runs the same experiment using their machines. We model
this situation by running Pylightnix with different storage settings.

\begin{pythontexcode}
Sa=mkSS('_storageA')
Sb=mkSS('_storageB')
fsinit(Sa,remove_existing=True)
fsinit(Sb,remove_existing=True)
rrefA=realize1(instantiate(stage_all, S=Sa))
rrefB=realize1(instantiate(stage_all, S=Sb))
print(rrefA)
print(rrefB)
\end{pythontexcode}

Our annealing problem is stochastic by nature so the results naturally differ.

\subsection{Synchronization}

Alice sends her results to Bob, so he faces a problem of picking the best
realizaion.

\begin{pythontexcode}
print("Bob's storage before the sync:")
for rref in allrrefs(S=Sb):
  print(rref)
arch=Path('archive.zip')
spack([rrefA], arch, S=Sa)
# .. Alice transfers the archive to Bob using her favorite pigeon post service.
sunpack(arch, S=Sb)
print("Bob's storage after the sync:")
for rref in allrrefs(S=Sb):
  print(rref)
\end{pythontexcode}

\mystdout


Bob writes a custom matcher that selects the realization which has the best
annealing result.

\begin{pythontexcode}
def match_min(S, rrefs:List[RRef])->List[RRef]:
  avail=[np.load(mklens(rref,S=S).trace_fs.syspath)[-1] for rref in rrefs]
  best=sorted(zip(avail,rrefs))[0]
  if best[1] in allrrefs(Sa):
    print(f"Picking Alice ({best[0]}) out of {avail}")
  else:
    print(f"Picking Bob ({best[0]}) out of {avail}")
  return [best[1]]

def stage_all2(r:Registry):
  ds=stage_params(r)
  cl=redefine(stage_anneal, new_matcher=match_min)(r=r,ref_params=ds)
  vis=stage_plot(r,ref_anneal=cl)
  return vis

rref_best=realize1(instantiate(stage_all2,S=Sb))
print(rref_best)
\end{pythontexcode}

\mystdout
