\begin{abstract}

Pylightnix is a Python domain specific language library for managing
filesystem-based immutable data objects, inspired by
\href{https://edolstra.github.io/pubs/phd-thesis.pdf}{Purely Functional
Software Deployment Model thesis by Eelco Dolstra} and the
\href{https://nixos.org}{Nix} package manager. In contrast to Nix, Pylightnix
is primarily focused on managing the data for computer science experiments.
Traditional use case of domain-specific package management, as well as other
\href{https://en.wikipedia.org/wiki/Blackboard_design_pattern}{blackboard
application use cases} are also supported.

With the help of Pylightnix API, applications
\begin{itemize}
  \item Store the data in form of linked immutable filesystem objects here
    called \textbf{stages}.
  \item Create (in our terms, \textbf{realize}) such objects, access its
    data, navigate through dependencies.
  \item Re-run realization algorithms whenever inputs change. Pylightnix
    may decide to re-create either a whole or a part of the stage object
    collection according to the changes in prerequisites.
  \item Manage the outcomes of \textbf{non-deterministic} stage realizations.
    As one example, users may define a Pylightnix stage to depend (in a
    user-defined sense) on top-10 random instances of a trained machine learning
    model.
\end{itemize}

\end{abstract}

