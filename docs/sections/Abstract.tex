\begin{abstract}

Pylightnix is a minimalistic Python library for managing filesystem-based
immutable data objects, inspired by
\href{https://edolstra.github.io/pubs/phd-thesis.pdf}{Purely Functional
Software Deployment Model thesis by Eelco Dolstra} and the
\href{https://nixos.org}{Nix} package manager.

The library may be thought as of low-level API for creating caching wrappers
for a subset of Python functions. In particular, Pylightnix allows user to

\begin{itemize}
  \item Prepare (or, in our terms, \textbf{instantiate}) the computation plan
    aimed at creating a tree of linked immutable stage objects, stored
    in the filesystem.
  \item Implement (\textbf{realize}) the prepared computation plan, access the
    resulting artifacts. Pylightnix is able to handle possible
    \textbf{non-deterministic} results of the computation. As one example, it is
    possible to define a stage depending on best top-10 instances (in a
    user-defined sence) of prerequisite stages.
  \item Handle changes in the computation plan, re-use the existing artifacts
    whenever possible.
  \item Gain full control over all aspects of the cached data including the
    garbage-collection.
\end{itemize}

\end{abstract}

