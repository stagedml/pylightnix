\section{Definitions}

\subsection{Notation}

Functions
\[ \exists f : A \to B \]
means that we claim that there exists a function \(f\) mapping items of set \(A\)
to items of set \(B\). We define sets by enumerating their items using the
standard \( \set{} \) notation.

Booleans
\[ \Bool = \set{ \True, \False } \]
Natural numbers plus zero
\[ \mathbb{Z} = \set{ 0,1,2,... } \]
Tuples (Products)
\[ (X,Y) = \set{ (x,y) ; x \in X \land y \in Y } \]
\[ (X,Y,Z,...) = (X,(Y,(Z,...))) \]
For every tuple there exist two unique functions:
\[ \exists  \pi_1 : (X,Y) \to X \]
\[ \exists  \pi_2 : (X,Y) \to Y \]
We access fields of nested tuples by using composition over these functions.
Often we want to define tuples together with accessors functions. By
saying
\[ T = (a:X,b:Y,c:Z,...) \]
we mean
\[ T = (X,(Y,(Z,...))) \land \]
\[ \exists a : T \to X = \pi_1 \land \exists b : T \to Y = \pi_1 \circ \pi_2
\land \exists c : T \to Z = \pi_1 \circ \pi_2 \circ \pi_2 \land ... \]
Unions (Co-products)
\[ X|Y = \set{u | u \in X \lor u \in Y }\]
Note: X and Y has to be separable from each other, i.e. for any particular \( u
\) it should be possible to tell whether it comes from X or from Y.
\hfill \break \noindent
Optional
\[ \mathbb{O}[X] = X | \oslash \]
Lists
\[ \List[X] = \mathbb{O}[(X,\mathbb{O}[(X,...)])] \]
Dictionaries
\[ \Dict[K,V] = \List[(K,V)] \]
\[ \exists item : \Dict[K,V] \to K \to \Opt[V] \]
Note: we assume that \(item \) returns the last \( V \) element with matching \(
K \).
\hfill \break \noindent


\subsection{Core entities}

TODO
