\section{Development}

\subsection{Environment setup}

While Pylightnix enjoys minimum run-time dependencies, it does require lots of
development dependencies to be installed.  We recomend to install
\href{https://nixos.org/nix}{Nix} package manager to install all these
dependencies automatically. There are following options available:

\begin{itemize}

  \item File \texttt{default.nix} contains the expression for building the
    end-user Python package. To run the build, execute \texttt{nix-build} in the
    project's root folder:

    \begin{shellcode}
    $ nix-build
    \end{shellcode}

    The result will appear under the symbolic link \texttt{./result} in the
    current folder.

  \item In \texttt{shell.nix} we define dependencies and specify configuration
    of the development shell. Use standard \texttt{nix-shell} tool to enter this
    shell:

    \begin{shellcode}
    $ nix-shell
    \end{shellcode}

    Inside the shell, one would use \texttt{GNU Make} with the following
    targets.

    \begin{itemize}
      \item \texttt{make tc} - Typechecks the project using MyPy
      \item \texttt{make coverage} and \texttt{make test} (a synonym) - Run
        Pylightnix test and prepare the coverage reports.
      \item \texttt{make docs} - Builds the Pylightnix API reference and Tex
        documentation.
      \item \texttt{make wheel} - Builds the wheel pip package
      \item \texttt{sudo -H make install} - For non-Nix systems: installs the
        package systemwide. This target should be called after a successful call
        to \texttt{make wheel}. Nix users should import \texttt{default.nix}
        where requred or use \texttt{nix-build}.
    \end{itemize}

    The below utilitary targets rely on private account credentials which are
    not included. One would need to adjust some settings in order to make it
    work.

    \begin{itemize}

      \item \texttt{make publish-docs} - Commits documentation PDFs to
        Author's docs repository. Change the repository URL in the Makefile
        before using this target.

      \item \texttt{make coverage-upload} Uploads the test coverage report to
        the Author's CodeCov account. There must be a file \texttt{.codecovrc}
        containing the CodeCov access token.

    \end{itemize}

\end{itemize}


\subsection{References}

\begin{itemize}
  \item Python:
    \begin{itemize}
      \item
        \href{https://packaging.python.org/guides/single-sourcing-package-version/}{Approaches}
        to versioning
      \item
        \href{https://pypi.org/project/setuptools-scm/}{Setuptool-scm} versioning
      \item
        \href{https://docs.python.org/3.7/distutils/sourcedist.html#manifest-template}{On Manifests}. Manifests seem to be useless in our case.
    \end{itemize}
\end{itemize}



