\section{Development}

\subsection{Environment setup}

The \href{https://nixos.org/nix}{Nix} package manager is required to setup the
development environment. While Pylightnix enjoys minimum run-time dependencies,
it does require lots of development dependencies to be installed. All the
requirements are encoded as Nix expressions:

\begin{itemize}

  \item File \texttt{default.nix} contains the expression to builds the final
    Python package from source. To run the build, execute in the project's root
    folder:

    \begin{shellcode}
    $ nix-build
    \end{shellcode}

    The result would appear under the symbolic link \texttt{./result} in the
    current folder.

  \item In \texttt{shell.nix} we specify how to prepare the development shell.
    To open the shell, execute

    \begin{shellcode}
    $ nix-shell
    \end{shellcode}

    Inside the development shell, one could use \texttt{GNU Make} to complete
    various development tasks. The most important Make targets are

    \begin{itemize}
      \item \texttt{make tc} - Typechecks the project using MyPy
      \item \texttt{make coverage} and \texttt{make test} (a synonym) - Run
        Pylightnix test and prepare the coverage reports.
      \item \texttt{make wheel} - Builds the wheel pip package
      \item \texttt{make docs} - Builds the Tex documentation, including this
        one.
      \item \texttt{make docs} - Builds the documentation, including this
        one.
    \end{itemize}

    Utilitary targets listed below rely on various sorts of private account
    information. One may want to change change settings to make those rules work
    for they.

    \begin{itemize}

      \item \texttt{make publish-docs} - Commits documentation PDFs to
        Author's docs repository. Change the repository URL in the Makefile
        before using this target.

      \item \texttt{make coverage-uploade} Uploads the test coverage report to
        the Author's CodeCov account. There must be a file \texttt{.codecovrc}
        containing the CodeCov access token.

    \end{itemize}

\end{itemize}


\subsection{References}

\begin{itemize}
  \item Python:
    \begin{itemize}
      \item
        \href{https://packaging.python.org/guides/single-sourcing-package-version/}{Approaches}
        to versioning
      \item
        \href{https://pypi.org/project/setuptools-scm/}{Setuptool-scm} versioning
      \item
        \href{https://docs.python.org/3.7/distutils/sourcedist.html#manifest-template}{On Manifests}. Manifests seem to be useless in our case.
    \end{itemize}
\end{itemize}



